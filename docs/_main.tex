% Options for packages loaded elsewhere
\PassOptionsToPackage{unicode}{hyperref}
\PassOptionsToPackage{hyphens}{url}
%
\documentclass[
]{book}
\usepackage{amsmath,amssymb}
\usepackage{lmodern}
\usepackage{iftex}
\ifPDFTeX
  \usepackage[T1]{fontenc}
  \usepackage[utf8]{inputenc}
  \usepackage{textcomp} % provide euro and other symbols
\else % if luatex or xetex
  \usepackage{unicode-math}
  \defaultfontfeatures{Scale=MatchLowercase}
  \defaultfontfeatures[\rmfamily]{Ligatures=TeX,Scale=1}
\fi
% Use upquote if available, for straight quotes in verbatim environments
\IfFileExists{upquote.sty}{\usepackage{upquote}}{}
\IfFileExists{microtype.sty}{% use microtype if available
  \usepackage[]{microtype}
  \UseMicrotypeSet[protrusion]{basicmath} % disable protrusion for tt fonts
}{}
\makeatletter
\@ifundefined{KOMAClassName}{% if non-KOMA class
  \IfFileExists{parskip.sty}{%
    \usepackage{parskip}
  }{% else
    \setlength{\parindent}{0pt}
    \setlength{\parskip}{6pt plus 2pt minus 1pt}}
}{% if KOMA class
  \KOMAoptions{parskip=half}}
\makeatother
\usepackage{xcolor}
\IfFileExists{xurl.sty}{\usepackage{xurl}}{} % add URL line breaks if available
\IfFileExists{bookmark.sty}{\usepackage{bookmark}}{\usepackage{hyperref}}
\hypersetup{
  pdftitle={797ML Handbook},
  pdfauthor={Steve Linberg},
  hidelinks,
  pdfcreator={LaTeX via pandoc}}
\urlstyle{same} % disable monospaced font for URLs
\usepackage{longtable,booktabs,array}
\usepackage{calc} % for calculating minipage widths
% Correct order of tables after \paragraph or \subparagraph
\usepackage{etoolbox}
\makeatletter
\patchcmd\longtable{\par}{\if@noskipsec\mbox{}\fi\par}{}{}
\makeatother
% Allow footnotes in longtable head/foot
\IfFileExists{footnotehyper.sty}{\usepackage{footnotehyper}}{\usepackage{footnote}}
\makesavenoteenv{longtable}
\usepackage{graphicx}
\makeatletter
\def\maxwidth{\ifdim\Gin@nat@width>\linewidth\linewidth\else\Gin@nat@width\fi}
\def\maxheight{\ifdim\Gin@nat@height>\textheight\textheight\else\Gin@nat@height\fi}
\makeatother
% Scale images if necessary, so that they will not overflow the page
% margins by default, and it is still possible to overwrite the defaults
% using explicit options in \includegraphics[width, height, ...]{}
\setkeys{Gin}{width=\maxwidth,height=\maxheight,keepaspectratio}
% Set default figure placement to htbp
\makeatletter
\def\fps@figure{htbp}
\makeatother
\setlength{\emergencystretch}{3em} % prevent overfull lines
\providecommand{\tightlist}{%
  \setlength{\itemsep}{0pt}\setlength{\parskip}{0pt}}
\setcounter{secnumdepth}{5}
\usepackage{booktabs}
\ifLuaTeX
  \usepackage{selnolig}  % disable illegal ligatures
\fi
\usepackage[]{natbib}
\bibliographystyle{plainnat}

\title{797ML Handbook}
\author{Steve Linberg}
\date{2022-04-03}

\begin{document}
\maketitle

{
\setcounter{tocdepth}{1}
\tableofcontents
}
\hypertarget{about}{%
\chapter{About}\label{about}}

This book is being written as part of a final project for 797ML at UMass
Amherst, spring 2022. It contains a simple reference and breakdown for a
couple of dozen core methods used in machine learning.

The intent is twofold:

\begin{enumerate}
\def\labelenumi{\arabic{enumi}.}
\tightlist
\item
  Serve as a reference for the basics of the material covered in the class, using language and examples that are as simple as possible to explain the core concepts and how to do them;
\item
  Force myself to learn these techniques better by carrying out the above.
\end{enumerate}

\hypertarget{simple-linear-regression}{%
\chapter{Simple Linear Regression}\label{simple-linear-regression}}

\hypertarget{tldr}{%
\section{TL;DR}\label{tldr}}

\begin{description}
\tightlist
\item[What it does]
Looks to see how well a single predictor variable predicts an outcome, like \emph{how well do years of education predict salary?}
\item[When to do it]
When you want to see if pretty much the simplest possible model provides enough of an explanation of variance for your purposes
\item[How to do it]
With the \texttt{lm()} function, among other ways
\item[How to assess it]
Look for a significant \(p\)-value for the predictor, and a reasonable \(R^2\)
\end{description}

\hypertarget{what-it-does}{%
\section{What it does}\label{what-it-does}}

Simple linear regression is where it all begins; among the simplest of all of the regression techniques in analysis, which attempts to estimate a slope and an intercept line for a set of observations using a single predictor variable \(X\) and an output variable \(Y\). It uses ordinary least squares (OLS) to build its model, looking for the line through the mean of \(X\) and \(Y\) that has the smallest sum of squares between the predicted and observed values.

\hypertarget{when-to-do-it}{%
\section{When to do it}\label{when-to-do-it}}

It is a simple first step for looking at data to see if there is an easy single-variable model that does a reasonable job predicting outcomes using one predictor variable. Sometimes, it can be good enough! It has the advantage of being easy to execute, to understand and to communicate, and the value of these factors should not be underestimated. Communicating with non-specialists is an important aspect of a data scientist's job.

Linear regression requires a dataset with a continuous outcome variable; it is easiest and most effective if the predictor variable is also numeric, whether continuous or discrete. It is possible to do linear regression with non-numeric predictors, such as true/false or ordered responses, by converting the predictors to a numeric scale.

\hypertarget{how-to-do-it}{%
\section{How to do it}\label{how-to-do-it}}

\hypertarget{how-to-interpret-the-output}{%
\section{How to interpret the output}\label{how-to-interpret-the-output}}

\hypertarget{where-to-learn-more}{%
\section{Where to learn more}\label{where-to-learn-more}}

\hypertarget{multiple-linear-regression}{%
\chapter{Multiple Linear Regression}\label{multiple-linear-regression}}

\hypertarget{tldr-1}{%
\section{TL;DR}\label{tldr-1}}

\begin{description}
\tightlist
\item[What it does]
Looks to see how well multiple predictor variables predict an outcome, like \emph{how well do years of education and age predict salary?}
\item[When to do it]
When a \protect\hyperlink{simple-linear-regression}{simple linear regression} doesn't provide a good enough explanation of variance, and you want to see if adding additional variables provides a better one
\item[How to do it]
With the \texttt{lm()} function, utilizing more than one predictor
\item[How to assess it]
Look for significant \(p\)-values for the predictors, and a reasonable adjusted-\(R^2\)
\end{description}

\hypertarget{what-it-does-1}{%
\section{What it does}\label{what-it-does-1}}

Multiple linear regression is the first natural extension of simple linear regression. It allows for more than one predictor variable to be specified. It is also possible to combine predictors in interactions, to find out if combinations of predictors have different effects than simply adding them to the model. XXX explain/demo

\hypertarget{when-to-do-it-1}{%
\section{When to do it}\label{when-to-do-it-1}}

Use multiple linear regression when a simple linear regression doesn't provide a good enough explanation of the variance you're observing, and you want to see if adding more predictors provides a better fit. Typically, this would be in response to either a low \(R^2\) that leaves a lot of unexplained variance, or even just a visual conclusion drawn from seeing a plot of a linear model with an unsatisfactory regression line.

\hypertarget{how-to-do-it-1}{%
\section{How to do it}\label{how-to-do-it-1}}

The \texttt{lm()} function, using more than one predictor in the formula.

\hypertarget{how-to-interpret-the-output-1}{%
\section{How to interpret the output}\label{how-to-interpret-the-output-1}}

\hypertarget{where-to-learn-more-1}{%
\section{Where to learn more}\label{where-to-learn-more-1}}

\hypertarget{logistic-regression}{%
\chapter{Logistic Regression}\label{logistic-regression}}

\hypertarget{tldr-2}{%
\section{TL;DR}\label{tldr-2}}

What it does
:

When to do it
:

How to do it
:

How to assess it
:

\hypertarget{what-it-does-2}{%
\section{What it does}\label{what-it-does-2}}

\hypertarget{when-to-do-it-2}{%
\section{When to do it}\label{when-to-do-it-2}}

\hypertarget{how-to-do-it-2}{%
\section{How to do it}\label{how-to-do-it-2}}

\hypertarget{how-to-interpret-the-output-2}{%
\section{How to interpret the output}\label{how-to-interpret-the-output-2}}

\hypertarget{where-to-learn-more-2}{%
\section{Where to learn more}\label{where-to-learn-more-2}}

\hypertarget{multiple-logistic-regression}{%
\chapter{Multiple Logistic Regression}\label{multiple-logistic-regression}}

\hypertarget{tldr-3}{%
\section{TL;DR}\label{tldr-3}}

What it does
:

When to do it
:

How to do it
:

How to assess it
:

\hypertarget{what-it-does-3}{%
\section{What it does}\label{what-it-does-3}}

\hypertarget{when-to-do-it-3}{%
\section{When to do it}\label{when-to-do-it-3}}

\hypertarget{how-to-do-it-3}{%
\section{How to do it}\label{how-to-do-it-3}}

\hypertarget{how-to-interpret-the-output-3}{%
\section{How to interpret the output}\label{how-to-interpret-the-output-3}}

\hypertarget{where-to-learn-more-3}{%
\section{Where to learn more}\label{where-to-learn-more-3}}

\hypertarget{linear-discriminant-analysis}{%
\chapter{Linear Discriminant Analysis}\label{linear-discriminant-analysis}}

\hypertarget{tldr-4}{%
\section{TL;DR}\label{tldr-4}}

What it does
:

When to do it
:

How to do it
:

How to assess it
:

\hypertarget{what-it-does-4}{%
\section{What it does}\label{what-it-does-4}}

\hypertarget{when-to-do-it-4}{%
\section{When to do it}\label{when-to-do-it-4}}

\hypertarget{how-to-do-it-4}{%
\section{How to do it}\label{how-to-do-it-4}}

\hypertarget{how-to-interpret-the-output-4}{%
\section{How to interpret the output}\label{how-to-interpret-the-output-4}}

\hypertarget{where-to-learn-more-4}{%
\section{Where to learn more}\label{where-to-learn-more-4}}

\hypertarget{quadratic-discriminant-analysis}{%
\chapter{Quadratic Discriminant Analysis}\label{quadratic-discriminant-analysis}}

\hypertarget{tldr-5}{%
\section{TL;DR}\label{tldr-5}}

What it does
:

When to do it
:

How to do it
:

How to assess it
:

\hypertarget{what-it-does-5}{%
\section{What it does}\label{what-it-does-5}}

\hypertarget{when-to-do-it-5}{%
\section{When to do it}\label{when-to-do-it-5}}

\hypertarget{how-to-do-it-5}{%
\section{How to do it}\label{how-to-do-it-5}}

\hypertarget{how-to-interpret-the-output-5}{%
\section{How to interpret the output}\label{how-to-interpret-the-output-5}}

\hypertarget{where-to-learn-more-5}{%
\section{Where to learn more}\label{where-to-learn-more-5}}

\hypertarget{naive-bayes}{%
\chapter{Naive Bayes}\label{naive-bayes}}

\hypertarget{tldr-6}{%
\section{TL;DR}\label{tldr-6}}

What it does
:

When to do it
:

How to do it
:

How to assess it
:

\hypertarget{what-it-does-6}{%
\section{What it does}\label{what-it-does-6}}

\hypertarget{when-to-do-it-6}{%
\section{When to do it}\label{when-to-do-it-6}}

\hypertarget{how-to-do-it-6}{%
\section{How to do it}\label{how-to-do-it-6}}

\hypertarget{how-to-interpret-the-output-6}{%
\section{How to interpret the output}\label{how-to-interpret-the-output-6}}

\hypertarget{where-to-learn-more-6}{%
\section{Where to learn more}\label{where-to-learn-more-6}}

\hypertarget{k-nearest-neighbors}{%
\chapter{K-Nearest Neighbors}\label{k-nearest-neighbors}}

\hypertarget{tldr-7}{%
\section{TL;DR}\label{tldr-7}}

What it does
:

When to do it
:

How to do it
:

How to assess it
:

\hypertarget{what-it-does-7}{%
\section{What it does}\label{what-it-does-7}}

\hypertarget{when-to-do-it-7}{%
\section{When to do it}\label{when-to-do-it-7}}

\hypertarget{how-to-do-it-7}{%
\section{How to do it}\label{how-to-do-it-7}}

\hypertarget{how-to-interpret-the-output-7}{%
\section{How to interpret the output}\label{how-to-interpret-the-output-7}}

\hypertarget{where-to-learn-more-7}{%
\section{Where to learn more}\label{where-to-learn-more-7}}

\hypertarget{poisson-regression}{%
\chapter{Poisson Regression}\label{poisson-regression}}

\hypertarget{tldr-8}{%
\section{TL;DR}\label{tldr-8}}

What it does
:

When to do it
:

How to do it
:

How to assess it
:

\hypertarget{what-it-does-8}{%
\section{What it does}\label{what-it-does-8}}

\hypertarget{when-to-do-it-8}{%
\section{When to do it}\label{when-to-do-it-8}}

\hypertarget{how-to-do-it-8}{%
\section{How to do it}\label{how-to-do-it-8}}

\hypertarget{how-to-interpret-the-output-8}{%
\section{How to interpret the output}\label{how-to-interpret-the-output-8}}

\hypertarget{where-to-learn-more-8}{%
\section{Where to learn more}\label{where-to-learn-more-8}}

\hypertarget{cross-validation}{%
\chapter{Cross-Validation}\label{cross-validation}}

\hypertarget{tldr-9}{%
\section{TL;DR}\label{tldr-9}}

What it does
:

When to do it
:

How to do it
:

How to assess it
:

\hypertarget{what-it-does-9}{%
\section{What it does}\label{what-it-does-9}}

\hypertarget{when-to-do-it-9}{%
\section{When to do it}\label{when-to-do-it-9}}

\hypertarget{how-to-do-it-9}{%
\section{How to do it}\label{how-to-do-it-9}}

\hypertarget{how-to-interpret-the-output-9}{%
\section{How to interpret the output}\label{how-to-interpret-the-output-9}}

\hypertarget{where-to-learn-more-9}{%
\section{Where to learn more}\label{where-to-learn-more-9}}

\hypertarget{bootstrap}{%
\chapter{Bootstrap}\label{bootstrap}}

\hypertarget{tldr-10}{%
\section{TL;DR}\label{tldr-10}}

What it does
:

When to do it
:

How to do it
:

How to assess it
:

\hypertarget{what-it-does-10}{%
\section{What it does}\label{what-it-does-10}}

\hypertarget{when-to-do-it-10}{%
\section{When to do it}\label{when-to-do-it-10}}

\hypertarget{how-to-do-it-10}{%
\section{How to do it}\label{how-to-do-it-10}}

\hypertarget{how-to-interpret-the-output-10}{%
\section{How to interpret the output}\label{how-to-interpret-the-output-10}}

\hypertarget{where-to-learn-more-10}{%
\section{Where to learn more}\label{where-to-learn-more-10}}

\hypertarget{best-subset-selection}{%
\chapter{Best Subset Selection}\label{best-subset-selection}}

\hypertarget{tldr-11}{%
\section{TL;DR}\label{tldr-11}}

What it does
:

When to do it
:

How to do it
:

How to assess it
:

\hypertarget{what-it-does-11}{%
\section{What it does}\label{what-it-does-11}}

\hypertarget{when-to-do-it-11}{%
\section{When to do it}\label{when-to-do-it-11}}

\hypertarget{how-to-do-it-11}{%
\section{How to do it}\label{how-to-do-it-11}}

\hypertarget{how-to-interpret-the-output-11}{%
\section{How to interpret the output}\label{how-to-interpret-the-output-11}}

\hypertarget{where-to-learn-more-11}{%
\section{Where to learn more}\label{where-to-learn-more-11}}

\hypertarget{stepwise-selection}{%
\chapter{Stepwise Selection}\label{stepwise-selection}}

\hypertarget{tldr-12}{%
\section{TL;DR}\label{tldr-12}}

What it does
:

When to do it
:

How to do it
:

How to assess it
:

\hypertarget{what-it-does-12}{%
\section{What it does}\label{what-it-does-12}}

\hypertarget{when-to-do-it-12}{%
\section{When to do it}\label{when-to-do-it-12}}

\hypertarget{how-to-do-it-12}{%
\section{How to do it}\label{how-to-do-it-12}}

\hypertarget{how-to-interpret-the-output-12}{%
\section{How to interpret the output}\label{how-to-interpret-the-output-12}}

\hypertarget{where-to-learn-more-12}{%
\section{Where to learn more}\label{where-to-learn-more-12}}

\hypertarget{ridge-regression}{%
\chapter{Ridge Regression}\label{ridge-regression}}

\hypertarget{tldr-13}{%
\section{TL;DR}\label{tldr-13}}

What it does
:

When to do it
:

How to do it
:

How to assess it
:

\hypertarget{what-it-does-13}{%
\section{What it does}\label{what-it-does-13}}

\hypertarget{when-to-do-it-13}{%
\section{When to do it}\label{when-to-do-it-13}}

\hypertarget{how-to-do-it-13}{%
\section{How to do it}\label{how-to-do-it-13}}

\hypertarget{how-to-interpret-the-output-13}{%
\section{How to interpret the output}\label{how-to-interpret-the-output-13}}

\hypertarget{where-to-learn-more-13}{%
\section{Where to learn more}\label{where-to-learn-more-13}}

\hypertarget{lasso}{%
\chapter{Lasso}\label{lasso}}

\hypertarget{tldr-14}{%
\section{TL;DR}\label{tldr-14}}

What it does
:

When to do it
:

How to do it
:

How to assess it
:

\hypertarget{what-it-does-14}{%
\section{What it does}\label{what-it-does-14}}

\hypertarget{when-to-do-it-14}{%
\section{When to do it}\label{when-to-do-it-14}}

\hypertarget{how-to-do-it-14}{%
\section{How to do it}\label{how-to-do-it-14}}

\hypertarget{how-to-interpret-the-output-14}{%
\section{How to interpret the output}\label{how-to-interpret-the-output-14}}

\hypertarget{where-to-learn-more-14}{%
\section{Where to learn more}\label{where-to-learn-more-14}}

\hypertarget{principal-component-regression}{%
\chapter{Principal Component Regression}\label{principal-component-regression}}

\hypertarget{tldr-15}{%
\section{TL;DR}\label{tldr-15}}

What it does
:

When to do it
:

How to do it
:

How to assess it
:

\hypertarget{what-it-does-15}{%
\section{What it does}\label{what-it-does-15}}

\hypertarget{when-to-do-it-15}{%
\section{When to do it}\label{when-to-do-it-15}}

\hypertarget{how-to-do-it-15}{%
\section{How to do it}\label{how-to-do-it-15}}

\hypertarget{how-to-interpret-the-output-15}{%
\section{How to interpret the output}\label{how-to-interpret-the-output-15}}

\hypertarget{where-to-learn-more-15}{%
\section{Where to learn more}\label{where-to-learn-more-15}}

\hypertarget{bagging}{%
\chapter{Bagging}\label{bagging}}

\hypertarget{tldr-16}{%
\section{TL;DR}\label{tldr-16}}

What it does
:

When to do it
:

How to do it
:

How to assess it
:

\hypertarget{what-it-does-16}{%
\section{What it does}\label{what-it-does-16}}

\hypertarget{when-to-do-it-16}{%
\section{When to do it}\label{when-to-do-it-16}}

\hypertarget{how-to-do-it-16}{%
\section{How to do it}\label{how-to-do-it-16}}

\hypertarget{how-to-interpret-the-output-16}{%
\section{How to interpret the output}\label{how-to-interpret-the-output-16}}

\hypertarget{where-to-learn-more-16}{%
\section{Where to learn more}\label{where-to-learn-more-16}}

\hypertarget{random-forests}{%
\chapter{Random Forests}\label{random-forests}}

\hypertarget{tldr-17}{%
\section{TL;DR}\label{tldr-17}}

What it does
:

When to do it
:

How to do it
:

How to assess it
:

\hypertarget{what-it-does-17}{%
\section{What it does}\label{what-it-does-17}}

\hypertarget{when-to-do-it-17}{%
\section{When to do it}\label{when-to-do-it-17}}

\hypertarget{how-to-do-it-17}{%
\section{How to do it}\label{how-to-do-it-17}}

\hypertarget{how-to-interpret-the-output-17}{%
\section{How to interpret the output}\label{how-to-interpret-the-output-17}}

\hypertarget{where-to-learn-more-17}{%
\section{Where to learn more}\label{where-to-learn-more-17}}

\hypertarget{boosting}{%
\chapter{Boosting}\label{boosting}}

\hypertarget{tldr-18}{%
\section{TL;DR}\label{tldr-18}}

What it does
:

When to do it
:

How to do it
:

How to assess it
:

\hypertarget{what-it-does-18}{%
\section{What it does}\label{what-it-does-18}}

\hypertarget{when-to-do-it-18}{%
\section{When to do it}\label{when-to-do-it-18}}

\hypertarget{how-to-do-it-18}{%
\section{How to do it}\label{how-to-do-it-18}}

\hypertarget{how-to-interpret-the-output-18}{%
\section{How to interpret the output}\label{how-to-interpret-the-output-18}}

\hypertarget{where-to-learn-more-18}{%
\section{Where to learn more}\label{where-to-learn-more-18}}

\hypertarget{bayesian-additive-regression-trees}{%
\chapter{Bayesian Additive Regression Trees}\label{bayesian-additive-regression-trees}}

\hypertarget{tldr-19}{%
\section{TL;DR}\label{tldr-19}}

What it does
:

When to do it
:

How to do it
:

How to assess it
:

\hypertarget{what-it-does-19}{%
\section{What it does}\label{what-it-does-19}}

\hypertarget{when-to-do-it-19}{%
\section{When to do it}\label{when-to-do-it-19}}

\hypertarget{how-to-do-it-19}{%
\section{How to do it}\label{how-to-do-it-19}}

\hypertarget{how-to-interpret-the-output-19}{%
\section{How to interpret the output}\label{how-to-interpret-the-output-19}}

\hypertarget{where-to-learn-more-19}{%
\section{Where to learn more}\label{where-to-learn-more-19}}

\hypertarget{support-vector-machines}{%
\chapter{Support Vector Machines}\label{support-vector-machines}}

\hypertarget{tldr-20}{%
\section{TL;DR}\label{tldr-20}}

What it does
:

When to do it
:

How to do it
:

How to assess it
:

\hypertarget{what-it-does-20}{%
\section{What it does}\label{what-it-does-20}}

\hypertarget{when-to-do-it-20}{%
\section{When to do it}\label{when-to-do-it-20}}

\hypertarget{how-to-do-it-20}{%
\section{How to do it}\label{how-to-do-it-20}}

\hypertarget{how-to-interpret-the-output-20}{%
\section{How to interpret the output}\label{how-to-interpret-the-output-20}}

\hypertarget{where-to-learn-more-20}{%
\section{Where to learn more}\label{where-to-learn-more-20}}

\hypertarget{principal-component-analysis}{%
\chapter{Principal Component Analysis}\label{principal-component-analysis}}

\hypertarget{tldr-21}{%
\section{TL;DR}\label{tldr-21}}

What it does
:

When to do it
:

How to do it
:

How to assess it
:

\hypertarget{what-it-does-21}{%
\section{What it does}\label{what-it-does-21}}

\hypertarget{when-to-do-it-21}{%
\section{When to do it}\label{when-to-do-it-21}}

\hypertarget{how-to-do-it-21}{%
\section{How to do it}\label{how-to-do-it-21}}

\hypertarget{how-to-interpret-the-output-21}{%
\section{How to interpret the output}\label{how-to-interpret-the-output-21}}

\hypertarget{where-to-learn-more-21}{%
\section{Where to learn more}\label{where-to-learn-more-21}}

\hypertarget{k-means-clustering}{%
\chapter{K-Means Clustering}\label{k-means-clustering}}

\hypertarget{tldr-22}{%
\section{TL;DR}\label{tldr-22}}

What it does
:

When to do it
:

How to do it
:

How to assess it
:

\hypertarget{what-it-does-22}{%
\section{What it does}\label{what-it-does-22}}

\hypertarget{when-to-do-it-22}{%
\section{When to do it}\label{when-to-do-it-22}}

\hypertarget{how-to-do-it-22}{%
\section{How to do it}\label{how-to-do-it-22}}

\hypertarget{how-to-interpret-the-output-22}{%
\section{How to interpret the output}\label{how-to-interpret-the-output-22}}

\hypertarget{where-to-learn-more-22}{%
\section{Where to learn more}\label{where-to-learn-more-22}}

\hypertarget{hierarchical-clustering}{%
\chapter{Hierarchical Clustering}\label{hierarchical-clustering}}

\hypertarget{tldr-23}{%
\section{TL;DR}\label{tldr-23}}

What it does
:

When to do it
:

How to do it
:

How to assess it
:

\hypertarget{what-it-does-23}{%
\section{What it does}\label{what-it-does-23}}

\hypertarget{when-to-do-it-23}{%
\section{When to do it}\label{when-to-do-it-23}}

\hypertarget{how-to-do-it-23}{%
\section{How to do it}\label{how-to-do-it-23}}

\hypertarget{how-to-interpret-the-output-23}{%
\section{How to interpret the output}\label{how-to-interpret-the-output-23}}

\hypertarget{where-to-learn-more-23}{%
\section{Where to learn more}\label{where-to-learn-more-23}}

  \bibliography{book.bib,packages.bib}

\end{document}
